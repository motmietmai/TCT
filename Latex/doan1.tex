\documentclass[12pt]{extreport}
\usepackage[utf8]{vietnam}
\usepackage[left=3.50cm, right=2.00cm, top=3.50cm, bottom=3.00cm]{geometry}
\usepackage{fancybox,graphicx}
\usepackage{mathrsfs} 
\usepackage{amsfonts}
\usepackage{longtable,array}
\usepackage{multirow}
\newlength\mylength
\newcolumntype{C}[1]{>{\centering\arraybackslash}p{#1}}
\usepackage[intlimits]{amsmath}
\usepackage{array}
\usepackage{algorithm}
\usepackage{algorithmicx}
\makeatletter
\renewcommand{\ALG@name}{Thuật toán}
\makeatother
\usepackage{algpseudocode}
\usepackage{amsxtra,amssymb,latexsym,amscd,amsthm}
\usepackage{tikz}
\usetikzlibrary{shapes.geometric}
\usetikzlibrary{positioning,automata}
\newtheorem{theorem}{Định lý}[chapter]
\newtheorem{definition}{Định nghĩa}[chapter]
\newtheorem{example}{Ví dụ}[chapter]
\newtheorem{lemma}[theorem]{Bổ đề}
%Tiêu đề
\usepackage{fancyhdr}
\pagestyle{fancy}
\lhead{}
\chead{}
\rhead{BÁO CÁO OTOMAT VÀ NGÔN NGỮ HÌNH THỨC}
\lfoot{}
\cfoot{\thepage}
\rfoot{}

\usepackage{xcolor}
\usepackage{listings}

\definecolor{mGreen}{rgb}{0,0.6,0}
\definecolor{mGray}{rgb}{0.5,0.5,0.5}
\definecolor{mPurple}{rgb}{0.58,0,0.82}
\definecolor{backgroundColour}{rgb}{0.95,0.95,0.92}

\lstdefinestyle{CStyle}{
	backgroundcolor=\color{backgroundColour},   
	commentstyle=\color{mGreen},
	keywordstyle=\color{magenta},
	numberstyle=\tiny\color{mGray},
	stringstyle=\color{mPurple},
	basicstyle=\footnotesize,
	breakatwhitespace=false,         
	breaklines=true,                 
	captionpos=b,                    
	keepspaces=true,                 
	numbers=left,                    
	numbersep=5pt,                  
	showspaces=false,                
	showstringspaces=false,
	showtabs=false,                  
	tabsize=2,
	language=C
}

\begin{document}

\thispagestyle{empty}
\thisfancypage{
	\setlength{\fboxsep}{0pt}
	\fbox}{}

\begin{center}
	
	{\fontsize{13pt}{1}\selectfont\textbf{TRƯỜNG ĐẠI HỌC BÁCH KHOA HÀ NỘI}}
	\\
	{\fontsize{13pt}{1}\selectfont\textbf{VIỆN TOÁN ỨNG DỤNG VÀ TIN HỌC}}
	\\		
	\textbf{--------------------  o0o  ---------------------}\\[1cm]
	\includegraphics[scale=0.2]{logo} \\[1.2cm]
	\textbf{{\large ĐỒ ÁN I}}
\textbf{}\\[1cm]
\textbf{{\large BÀI TOÁN NHẬN DIỆN KHUÔN MẶT}}\\[0.2cm]
\end{center}
\begin{flushleft}
\hspace{1.5 cm} \textbf{ Giáo viên hướng dẫn:\hspace{0.2cm}{ Ts. TRẦN NGỌC THĂNG }}\\[0.2cm]
\hspace{1.5 cm} \textbf{ Sinh viên thực hiện\hspace{0.3cm}:\hspace{0.2cm}{ TRỊNH HOÀNG ĐỨC\\
\hspace{6.6 cm}PHẠM NGỌC QUANG ANH\\
\hspace{6.6 cm}ĐẶNG HỮU TÚ}}\\[0.2cm]
\hspace{1.5 cm} \textbf{ Lớp\hspace{3.8cm}:\hspace{0.2cm}{ KSTN TOÁN TIN - K60}}\\
\end{flushleft}

\vspace{1.0cm}
\begin{center}
\textbf{{\large HÀ NỘI - 2019}}\\
\end{center}

\tableofcontents

\newpage
\chapter*{Mở đầu}
aa

\newpage
\chapter{Mô hình nhận diện khuôn mặt }
\section{Detect face}
aaa
\subsection{CNN}
aa
\subsection{MT-CNN}
aa

\section{Recognize face}
aa
\chapter{Kết quả}
aa
\newpage
\chapter{Kết luận}
aaa

\newpage
\begin{thebibliography}{12}
	\addcontentsline{toc}{chapter}{\quad\  \bf Tài liệu tham khảo}
	\bibitem{1}On Certain Formal Properties of Grammars\\
	\textit{Noam Chomsky. In Information and Control 2, page 137 - 167, 1959.}
	\bibitem{2} Otomat và ngôn ngữ hình thức\\
	\textit{Đoàn Văn Ban, Bộ môn Khoa học máy tính, Khoa Công nghệ thông tin, Đại học Thái Nguyên.}
	\bibitem{3}Introduction to the Theory of Computation\\
	\textit{Michael Sipser. Cengage Learning, Boston, third edtion, 2013.
	}
	\bibitem{4}A Second Course in Formal Language\\
	\textit{Jeffrey Shallit. Cambridge University Press, 2009.}
	
	
	
\end{thebibliography}

\end{document}